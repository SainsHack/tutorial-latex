%\documentclass{beamer}
\documentclass[envcountsect]{beamer}

\usepackage{multimedia}


\mode<presentation>
{
% \usetheme{Berlin}
  % or ...
%\usetheme{default}
%\usetheme{PaloAlto}
%\usetheme{Malmoe}
%\usetheme{Warsaw}
  \setbeamercovered{transparent}
  % or whatever (possibly just delete it)
\usecolortheme{wolverine}
%\usefonttheme{serif}
%\setbeamertemplate{blocks}[rounded][shadow=true]
%\usecolortheme{sidebartab}

%\setbeamercolor{some beamer element}{fg=red}
}
\beamertemplateballitem

\usepackage{color}
%\usepackage[english]{babel}
% or whatever

%\usepackage[latin1]{inputenc}
% or whatever

%\usepackage{helvetica}
\usepackage{accents}
\usepackage{hyperref}

\usepackage{times}
\usepackage[T1]{fontenc}
% Or whatever. Note that the encoding and the font should match. If T1
% does not look nice, try deleting the line with the fontenc.

\newcommand{\ul}{\underline}
\newcommand{\vx}{\bf x}
\newcommand{\la}{\lambda}
\newcommand{\e}{{\rm e}}


\newtheorem{prop}[theorem]{Proposition}
\newtheorem{quest}{Question}
\newtheorem{remark}{Remark}



\title
{Presentation with sound \newline Example for Beamer}
\author[]{Tatiana}
\institute[University of Leicester]
{University of Leicester}


\beamerdefaultoverlayspecification{<+->}


\begin{document}

\begin{frame}
%\sound[autostart, inlinesound]{}{show.wav}
  \titlepage
  \transduration{5}
\end{frame}

\begin{frame}
\frametitle{Structure }
\tableofcontents
\transduration{5}
%\sound[autostart, start=10.5s, inlinesound]{}{show.wav}
\end{frame}


\section{1.Sound}
\begin{frame}
\frametitle{Insert file as sound}
\transwipe \transduration{10}
\sound[autostart, start=5s, duration=15s, inlinesound]{}{show.wav}
Sound file can be easily added to every slide of the presentation using command {\it $\setminus$sound}.\\
see Manual page 130\\

\vspace{10pt}
It can be played from the very beginning of the slide...
\end{frame}

\begin{frame}
\frametitle{Sound}
\transblindsvertical
\transduration{30}


... or it can be played if {\it button} (below) is pressed...\\
\vspace{10pt}

\sound[start=5s, inlinesound]{{ \it button} }{show.wav}\\


\end{frame}

\begin{frame}
\frametitle{Sound} \transblindsvertical \transduration{40}
\sound[label=show1, inlinesound]{}{show.wav}\\

... or it can be played using {\it Beamer button} and {\it $\setminus$hyperlink} command.\\
\hyperlinkmovie[start=5s]{show1}{\beamerbutton{Play Sound}} (press it)\\

 \vspace{10pt}
 Or you can keep control in your hands and define where
to play sound and duration...
\transwipe \transduration{20}
\end{frame}
\begin{frame}
\frametitle{Sound} \transblindshorizontal
\begin{theorem}
It is possible to play sound piece-wise
\begin{enumerate}
\item<1-> present the 1st piece;
    \only<1>{\sound[autostart,start=1s,duration=18s,inlinesound]{}{paganini.wav}
    \transduration{18}}
\item<2-> no sound on this piece;
    \only<2>{\transduration{10}}
\item<3-> present the last piece
\only<3>{\sound[autostart,start=18.5s,inlinesound]{}{paganini.wav}
    \transduration{20}}
\end{enumerate}
\end{theorem}
\only<4>{
And, finally, add some more comments
\sound[autostart,inlinesound]{}{paganini.wav}
\transduration{40}
}
\end{frame}


\section{2. Movie}

\begin{frame}
\frametitle{Insert sound as movie}
\transglitter \movie[label=show,
start=5s]{}{show.wav}
Sound file can be added using {\it $\setminus$movie} command.\\
 See Page 125 for more options.\\

In this case it is possible to control  the file with buttons\\

\hyperlinkmovie[play]{show}{\beamerbutton{play}}
\hyperlinkmovie[pause]{show}{\beamerbutton{pause}}
\hyperlinkmovie[resume]{show}{\beamerbutton{resume}}
\hyperlinkmovie[stop]{show}{\beamerbutton{stop}}

but you need more time for this slide to be on the screen\\
{\it $\setminus$transduration$\{60\}$}  \\

  ( This slide is staying on the screen for  60 seconds)

\transduration{60}

\end{frame}

\begin{frame}
\frametitle{Final}
\transglitter
\movie[autostart,start=64s, duration=13s]{}{show.wav}

\center{\large{show must go on...} \transduration{15}}
\sound[inlinesound,encoding=signed,channels=2]{Play the sound!}{clean_speech.wav}
\end{frame}

\end{document}
